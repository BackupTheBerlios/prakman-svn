\chapter{Installation und Test}
\label{chap:InstallationUndTest}

\section{Testverfahren und Testdokumente mit automatischen Tests}
\label{sec:Testverfahren}
Die Software wurde auf unterschiedliche Arten auf korrekte Funktionsweise �berpr�ft. Zuerst individuell durch den Entwickler, der seine Komponente direkt nach Fertigstellung pr�ft. Anschlie�end durch die anderen Entwickler, die diese Komponente im Zusammenspiel mit ihren eigenen testen. Schlie�lich ein Performance-Test durch eine vorher programmierte Automatik. 

\subsection{Individuelle Komponenten-Tests}
\label{subsec:IndividuelleKomponenten-Tests}
Jeder Programmierer testet seine neu erstellte oder ge�nderte Komponente selbstst�ndig auf korrekte Funktionsweise, um von Anfang an interne Fehler der Komponenten ausschlie�en zu k�nnen, wenn sie im Zusammenspiel mit dem Rest des Programms getestet wird.

\subsection{Tests des Komponenten-Zusammenspiels}
\label{subsec:Komponenten-Zusammenspiel-Tests}
Dieses Tests wurden durchgef�hrt, wenn gr��ere Abschnitte des Programms fertiggestellt sind bzw. neue Komponenten zum bereits funktionierenden Hauptteil hinzugef�gt wurden. Dabei wurde sichergestellt, dass die neue Komponenten richtig mit den anderen zusammenarbeitet aber auch keine neuen Fehler mit sich bringt.

\subsection{Performance-Test}
\label{subsec:Performance-Test}
Beim automatisierten Test wurden 15.000 zuf�llig generierte Studentendaten in das Programm eingef�gt. Dabei lie� sich ``PrakMan'' weiterhin gut nutzen und blieb in seiner Funktionsweise stabil. Da diese Anzahl das normal erwartete Datenaufkommen von etwa 1.000 - 5.000 Studenten deutlich �bersteigt, gilt eine gute Funktionsweise als gesichert.\\
Der Test kann selbstst�ndig durchgef�hrt, indem der Konfigurationsdatei ``config.conf'' im gew�hlten Arbeitsverzeichnis der Eintrag ``CrashTest=<Anzahl einzuf�gender Testwerte>'' hinzugef�gt wird. Beim n�chsten Start wird dann die angegebene Anzahl an Testwerten der Datenbank hinzugef�gt.


\section{�bergabedokument}
\label{sec:Uebergabedokument}
�bergeben wird eine CD mit dem Programm als JAR-Datei, den .java Dateien, der
Dokumentation als PDF-Dokument sowie Javadoc und allen MS Visio-Dateien.
