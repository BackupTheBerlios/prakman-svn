%\glossary{name={Event},description={Eine Veranstaltung oder ein Praktikum f�r eine gegebene Veranstaltung}}
%\glossary{name={Project},description={Eine Aufgabe, die an Studierende verteilt werden kann. Projekte sind notwendig, um Noten zu verteilen. Ein Projekt kann demnach auch das Bestehen der gesamten Veranstaltung darstellen.}}

\abbrev{Event}{Eine Veranstaltung oder ein Praktikum f�r eine gegebene Veranstaltung}

\abbrev{PrakMan}{Steht f�r Praktikums-Manager, das dokumentierte Programm.}

\abbrev{Project}{Eine Aufgabe, die an Studierende verteilt werden kann. Projekte sind notwendig, um Noten zu verteilen. Ein Projekt kann demnach auch das Bestehen der gesamten Veranstaltung darstellen.}

\abbrev{SubVersion (SVN)}{SubVersion ist eine Open-Source-Software zur Versionsverwaltung von Dateien und Verzeichnissen.}

%\chapter{Begriffserkl�rung}
\label{chap:Glossar}

\printnomenclature
\clearpage

% Nomenklatur erstellen:

% "c:\Programme\MiKTeX 2.5\miktex\bin\makeindex.exe" index.nlo -s nomencl.ist -o index.nls

% -------------------
% Damit das Glossar richtig erstellt wird, m�ssen folgende Befehle aufgerufen werden:
% 1. Latex-Konvertierung
% 2. C:\Dokumente und Einstellungen\PR\Desktop\SE-Hausarbeit\Doku\PrakMan_Doku>"c:\Programme\MiKTeX 2.5\miktex\bin\makeindex.exe" -s index.ist -t index.glg -o index.gls index.glo
% 3. Latex-Konvertierung
