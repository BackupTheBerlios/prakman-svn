\chapter{Programmkenndaten}
\label{cha:Programmkenndaten}

\section{Programmname}
\label{sec:Programmname}
Das Programm tr�gt den Namen seiner Funktionsbestimmung: Praktikumsmanager, abgek�rzt ``PrakMan''.

\section{Versionsnummer und Freigabedatum}
\label{sec:Version}
Das Programm liegt aktuell in der Version 1.0 vor und ist seit dem 17.09.2007 zur Nutzung freigegeben.

\section{Mitarbeiter}
\label{sec:Bearbeiter}
Mitarbeiter, die zu gleichen Teilen zur Fertigstellung der aktuell vorliegenden Version beigetragen haben sind:

\begin{itemize}
\item Andreas Depping \hfill Andreas.Depping@fh-osnabrueck.de\\
Bramscher Stra�e 158 \hfill Telefon: 0541/98252103\\
49088 Osnabr�ck \hfill Mat-Nr. \#325677
\item Christian Lins \hfill Christian.Lins@fh-osnabrueck.de\\
Limberger Stra�e 102 \hfill Telefon: 0541/3342152\\
49080 Osnabr�ck \hfill Mat-Nr. \#322141
\item Kai Ritterbusch \hfill Kai.Ritterbusch@fh-osnabrueck.de\\
Beethovenstra�e 8 \hfill Telefon: 05406/819606\\
49191 Belm \hfill Mat-Nr. \#327879
\item Philipp Rollwage \hfill Philipp.Rollwage@fh-osnabrueck.de\\
Neulandstra�e 6 \hfill Telefon: 0541/3247040\\
49084 Osnabr�ck \hfill Mat-Nr. \#329302
\end{itemize}

% 1.4. Aufgabenbeschreibung / Pflichtenheft
\section{Pflichtenheft}
\label{sec:Pflichtenheft}
Pflichtenheft zur Erstellung einer Software zur Praktikaverwaltung.
Das Projekt Praktika-Verwaltung beinhaltet die Planung und Realisierung eines Java-Programmsystems zur Verwaltung von Lehrveranstaltungen an der FH Osnabr�ck.

\subsection{Zielbestimmung}
\label{subsec:Zielbestimmung}
Es sollen Praktika (zum Beispiel zur Lehrveranstaltung Software Engineering) verwaltet werden. Dazu ist es notwendig, ca. 60 Studierende mit Hilfe einer GUI zu erfassen, in Praktikumsgruppen zu jeweils ca. 20 aufzuteilen und deren Anwesenheit in einer der ca. 15 Veranstaltungen im Semester zu erfassen.
Au�erdem sollen die Studierenden Projekten zugeordnet werden k�nnen. Dazu m�ssen ca. 6 Projekte zum Teil mehrfach angelegt werden k�nnen, denen jeweils ca. 1-8 Studierende zugeordnet werden.
Au�erdem sollen den Studierenden Hausarbeiten zugeordnet werden k�nnen. Dazu m�ssen ca. 4-8 Hausarbeiten zum Teil mehrfach angelegt werden k�nnen, denen jeweils ca. 1-8 Studierende zugeordnet werden.
Die Anwendung soll unterschiedliche Listen in unterschiedlicher Sortierung drucken k�nnen:
\begin{itemize}
\item Gesamtliste aller Studierenden der Veranstaltung
\item Liste der Studierenden je Praktikum
\item Liste mit Anwesenheit (gesamt und je Praktikum)
\item Notenlisten (``bestanden'' f�r Projekt, Note f�r Hausarbeit)
\end{itemize}

Das Programm muss �ber eine persistente Datenhaltung verf�gen, d.h. die Daten m�ssen in einer Datenbank oder in CSV-Dateien gehalten werden k�nnen.
Die CSV-Dateien sollen mit Excel eingelesen werden k�nnen.
Von jedem Studierenden werden die �blichen Daten erfasst (Name, Mat.-Nr, Semester, Studiengang, Email). W�nschenswert w�re eine Daten�bernahme aus den von StudIP exportierten Teilnehmerdaten.
Insgesamt soll eine benutzerfreundliche und leicht erweiterbare Anwendung zur Unterst�tzung von Lehrenden entstehen.

\subsubsection{Musskriterien}
\label{subsubsec:Musskriterien}
\begin{itemize}
\item Drucken von Anwesenheitslisten, Notenlisten
\item Wechsel von Studierenden zwischen Praktika
\item M�glichkeit, Studierende aus Praktika zu entfernen
\item Sortierungen nach unterschiedlichen Kriterien
\end{itemize}

\subsubsection{Wunschkriterien}
\label{subsubsec:Wunschkriterien}
\begin{itemize}
\item Schnittstelle zu StudIP: Import der dort exportierten Teilnehmerdaten
\item Verwaltung eines Fotos je Teilnehmer
%\item Filtern von Objekten
%\item Druckvorschau
\end{itemize}

\subsubsection{Abgrenzungskriterien}
\label{subsubsec:Abgrenzungskriterien} 
\begin{itemize}
\item Das Programm stellt keinen Ersatz f�r StudIP dar, sondern eine Unterst�tzung.
\item Es soll keine direkte Anbindung an die StudIP-Datenbank geben.
\end{itemize}

\subsection{Produkteinsatz}
\label{subsec:Produkteinsatz}
Das Programm soll von Professoren oder Tutoren, welche mit der Leitung von Praktikumsgruppen betraut sind, genutzt werden.

\subsubsection{Anwendungsbereiche}
\label{subsubsec:Anwendungsbereiche}
Das Programm soll es Professoren bzw. Tutoren erleichtern, ihre Praktikumsgruppen einzuteilen, Aufgaben und Noten zu verteilen und auf all diese Informationen in einer �bersichtlichen Oberfl�che zugreifen zu k�nnen.
Zur Haltung der Daten kann dabei eine eigene, lokale Datenbank genutzt werden, oder auch eine externe, auf die mehrere Tutoren mit ``PrakMan'' zugreifen und gemeinsam ihre Daten verwalten.

\subsubsection{Zielgruppen}
\label{subsubsec:Zielgruppen}
Das Programm soll von Professoren oder wissenschaftlichen Mitarbeitern, prim�r der Fachhochschule Osnabr�ck, eingesetzt werden. M�glich w�re aber auch ein Betrieb in anderen Hochschulen.
Vorausgesetzt werden grobe Kenntnisse im Bereich von Datenbanken (Zugangsdaten) oder Allgemeinwissen �ber Programme (M�gliche Optionen �ber Rechtsklick erreichbar etc.).

\subsubsection{Betriebsbedingungen}
\label{subsubsec:Betriebsbedingungen}
Das Programm ben�tigt keine gesonderte Beaufsichtigung, sondern kann direkt auf jedem PC, der die auf Seite \pageref{sec:Anforderungen} genannten Voraussetzungen erf�llt, genutzt werden.

\subsubsection{Software}
\label{subsubsec:Software}
Siehe Punkt \ref{sec:Anforderungen} auf Seite \pageref{sec:Anforderungen}.

\subsubsection{Hardware}
\label{subsubsec:Hardware}
Das Programm erfordert keine besondere Hardware.

\subsubsection{Orgware}
\label{subsubsec:Orgware}
Um den zu organisierenden Veranstaltungen Studenten hinzuf�gen zu k�nnen, m�ssen Daten �ber diese bekannt sein. Dazu geh�ren die Matrikelnummer, Vorname, Nachname und eine E-Mail-Adresse. Diese Daten sollten im Voraus entweder auf Papier oder per CSV (f�r den Import) organisiert werden.

\subsection{Schnittstellen}
\label{subsec:Schnittstellen}
Das Programm bietet eine JDBC-Schnittstelle, �ber die unterschiedliche Datenbanken angebunden werden k�nnen. ``PrakMan'' unterst�tzt momentan HSQL und PostgreSQL.

\subsection{Produktfunktionen}
\label{subsec:Produktfunktionen}

\subsubsection{/F0001/Verbindung zur Datenbank}
\label{subsubsec:/F0001/}
\paragraph{Voraussetzung:}
Keine
\paragraph{Ablauf:}
\begin{enumerate}
\item Der Benutzer startet das Programm und w�hlt einen Datenbanktypen aus: Lokal (HSQL) oder PostgreSQL.
\end{enumerate}
\paragraph{Alternativen:}
\begin{enumerate}
\item[1a.] Die Verbindung schl�gt fehl. Das kann an einer nicht vorhandenen Internetverbindung oder einem nicht gestarteten Datenbank-Server liegen.
\end{enumerate}

\subsubsection{/F0002/Erstellen einer Veranstaltung}
\label{subsubsec:/F0002/}
\paragraph{Voraussetzung:}
\begin{enumerate}
\item Der Benutzer hat PrakMan gestartet und die Verbindung zu einer Datenbank erfolgreich hergestellt.
\end{enumerate}
\paragraph{Ablauf:}
\begin{enumerate}
\item Der Benutzer klickt im Baum mit der rechten Maustaste auf den Ast ``Veranstaltungen'' und w�hlt ``Hinzuf�gen'' aus.
\item Ein Professor wird ausgew�hlt, der die Veranstaltung betreuen soll.
\item Die neue Veranstaltung ist erstellt, nun k�nnen weitere Einstellungen getroffen werden. 
\end{enumerate}
\paragraph{Alternativen:}
\begin{enumerate}
\item[1a.] Ein Datenbankfehler tritt auf. Die Verbindung zur Datenbank sollte �berpr�ft werden.
\end{enumerate}

\subsubsection{/F0003/Hinzuf�gen von Studenten zu einer Veranstaltung}
\label{subsubsec:/F0003/}
\paragraph{Voraussetzung:}
\begin{enumerate}
\item Der Benutzer hat PrakMan gestartet und die Verbindung zu einer Datenbank erfolgreich hergestellt.
\item Es existiert mindestens eine Veranstaltung.
\end{enumerate}
\paragraph{Ablauf:}
\begin{enumerate}
\item Der Benutzer �ffnet die Ansicht der Veranstaltung �ber einen Doppelklick auf Selbige.
\item �ber den Plus-Button am unteren Rand des Fensters wird ein neues Panel ge�ffnet, �ber welches Studenten ausgew�hlt und durch einen Druck auf ``Ok'' der Veranstaltung hinzugef�gt werden.
\end{enumerate}
\paragraph{Alternativen:}
\begin{enumerate}
\item[2a.] Es existieren keine Studenten. In diesem Fall m�ssen erst neue Studenten manuell erstellt oder �ber CSV eingelesen werden.
\end{enumerate} 

\subsubsection{/F0004/Einteilung der Studenten in Gruppen}
\label{subsubsec:/F0004/}
\paragraph{Voraussetzung:}
\begin{enumerate}
\item Der Benutzer hat PrakMan gestartet und die Verbindung zu einer Datenbank erfolgreich hergestellt.
\item Es existiert mindestens eine Veranstaltung, deren Gruppen-Ansicht ge�ffnet ist.
\end{enumerate}
\paragraph{Ablauf:}
\begin{enumerate}
\item Ein Klick auf den Plus-Button erstellt eine neue Gruppe. �ber einen Doppelklick auf ihren Namen kann deren Beschreibung editiert werden.
\item Der Benutzer wechselt in die Teilnehmer-Ansicht und markiert dort �ber einfaches Ziehen mit der Maus, oder Shift/Strg-Klick einen oder mehrere Studenten, die zusammen eine Gruppe bilden sollen. �ber einen rechtsklick auf die Tabelle kann diesen markierten Studenten nun eine Gruppe zugewiesen werden.
\end{enumerate}
\paragraph{Alternativen:}
\begin{enumerate}
\item[1a.] Ein Datenbankfehler tritt auf. Die Verbindung zur Datenbank sollte �berpr�ft werden.
\item[2a.] Es existieren noch keine Studenten in der Veranstaltung. In diesem Fall, k�nnen sie einfach �ber den Plus-Button in der Teilnehmer-Ansicht hinzugef�gt werden.
\end{enumerate}

\subsubsection{/F0005/Ein Projekt hinzuf�gen}
\label{subsubsec:/F0005/}
\paragraph{Voraussetzung:}
\begin{enumerate}
\item Der Benutzer hat PrakMan gestartet und die Verbindung zu einer Datenbank erfolgreich hergestellt.
\item Es existiert mindestens eine Veranstaltung, deren Projekte-Ansicht ge�ffnet ist.
\end{enumerate}
\paragraph{Ablauf:}
\begin{enumerate}
\item Ein Klick auf den Plus-Button erstellt ein neues unbenanntes Projekt.
\item �ber einen Doppelklick auf Selbiges, k�nnen diesem Projekt Teilnehmer hinzugef�gt und au�erdem dessen Beschreibung editiert werden. Zus�tzlich kann eine ``Deadline'' gesetzt werden, zu welcher dieses Projekt beendet sein muss.
\end{enumerate}
\paragraph{Alternativen:}
\begin{enumerate}
\item[1a.] Ein Datenbankfehler tritt auf. Die Verbindung zur Datenbank sollte �berpr�ft werden.
\item[2a.] Es existieren keine Studenten, die zugeordnet werden k�nnen. In diesem Fall, k�nnen sie einfach �ber den Plus-Button in der Teilnehmer-Ansicht hinzugef�gt werden.
\end{enumerate}

\subsubsection{/F0006/Einen Termin hinzuf�gen}
\label{subsubsec:/F0006/}
\paragraph{Voraussetzung:}
\begin{enumerate}
\item Der Benutzer hat PrakMan gestartet und die Verbindung zu einer Datenbank erfolgreich hergestellt.
\item Es existiert mindestens eine Veranstaltung, deren Termine-Ansicht ge�ffnet ist.
\end{enumerate}
\paragraph{Ablauf:}
\begin{enumerate}
\item Ein Klick auf den Plus-Button �ffnet ein Auswahlfenster, in dem ein Termin f�r die Veranstaltung ausgew�hlt werden kann. Zuerst sollte eine Uhrzeit angegeben werden. Ein Klick auf einen Tag im Kalender f�gt diesen Termin dann der Veranstaltung hinzu.
\end{enumerate}
\paragraph{Alternativen:} Keine

\subsubsection{/F0007/Den Tutor einer Veranstaltung �ndern}
\label{subsubsec:/F0007/}
\paragraph{Voraussetzung:}
\begin{enumerate}
\item Der Benutzer hat PrakMan gestartet und die Verbindung zu einer Datenbank erfolgreich hergestellt.
\item Es existiert mindestens eine Veranstaltung.
\end{enumerate}
\paragraph{Ablauf:}
\begin{enumerate}
\item �ber einen Doppelklick auf eine Veranstaltung links im Baum, �ffnet sich deren Hauptansicht. Nun kann �ber die Dropdown-Box im oberen rechten Bereich einer der in der Datenbank registrierten Professoren ausgew�hlt werden.
\item �ber den Button ``Speichern'' (oben links) wird die Information in die Datenbank �bernommen.
\end{enumerate}
\paragraph{Alternativen:}
\begin{enumerate}
\item[1a.] Statt einer Veranstaltung wird im Baum ein Lehrender ausgew�hlt. Bei diesem kann dann �ber den Plus-Button eine Veranstaltung zugeordnet werden.
\item[1b.] Es ist kein weiterer Professor vorhanden. In dem Fall m�ssen manuell oder per CSV-Import neue Lehrende hinzugef�gt werden.
\item[2a.] Ein Datenbankfehler tritt auf. Die Verbindung zur Datenbank sollte �berpr�ft werden.
\end{enumerate}

\subsubsection{/F0008/Einen neuen Dozenten hinzuf�gen}
\label{subsubsec:/F0008/}
\paragraph{Voraussetzung:}
\begin{enumerate}
\item Der Benutzer hat PrakMan gestartet und die Verbindung zu einer Datenbank erfolgreich hergestellt.
\end{enumerate}
\paragraph{Ablauf:}
\begin{enumerate}
\item Der Benutzer klickt im Baum mit der rechten Maustaste auf den Ast ``Lehrende'' und w�hlt ``Hinzuf�gen'' aus.
\item Im auftauchenden Panel werden die ben�tigten Daten ``Vorname'' und ``Nachname'' eingetragen.
\item Ein Druck auf den Button ``Speichern'' �bertr�gt die Informationen in die Datenbank, der neue Dozent ist erstellt.
\end{enumerate}
\paragraph{Alternativen:}
\begin{enumerate}
\item[3a.] Ein Datenbankfehler tritt auf. Die Verbindung zur Datenbank sollte �berpr�ft werden.
\end{enumerate}

\subsubsection{/F0009/Einen neuen Studenten hinzuf�gen}
\label{subsubsec:/F0009/}
\paragraph{Voraussetzung:}
\begin{enumerate}
\item Der Benutzer hat PrakMan gestartet und die Verbindung zu einer Datenbank erfolgreich hergestellt.
\end{enumerate}
\paragraph{Ablauf:}
\begin{enumerate}
\item Der Benutzer klickt im Baum mit der rechten Maustaste auf den Ast ``Studenten'' und w�hlt ``Hinzuf�gen'' aus.
\item Im auftauchenden Eingabefeld wird die 6-stellige Matrikelnummer des Studenten angegeben. Ein Druck auf ``OK'' l�sst ein neues Panel erscheinen.
\item Im erschienenen Panel werden die ben�tigten Daten ``Vorname'', ``Nachname'' und ``E-Mail'' eingetragen.
\item Ein Druck auf den Button ``Speichern'' �bertr�gt die Informationen in die Datenbank, der neue Student ist erstellt.
\end{enumerate}
\paragraph{Alternativen:}
\begin{enumerate}
\item[4a.] Ein Datenbankfehler tritt auf. Die Verbindung zur Datenbank sollte �berpr�ft werden.
\end{enumerate}

\subsubsection{/F0010/Daten in CSV exportieren}
\label{subsubsec:/F0010/}
\paragraph{Voraussetzung:}
\begin{enumerate}
\item Der Benutzer hat PrakMan gestartet und die Verbindung zu einer Datenbank erfolgreich hergestellt.
\end{enumerate}
\paragraph{Ablauf:}
\begin{enumerate}
\item Der Benutzer klickt im Baum mit der rechten Maustaste auf einen der �ste ``Lehrende'', ``Studenten'' \emph{oder} direkt auf eine bestimmte Veranstaltung und w�hlt ``Export''.

Wird ein Ast ausgew�hlt, werden nur die Informationen der Objekte in diesem Ast in der CSV-Datei gespeichert. Wird eine einzelne Veranstaltung ausgew�hlt, werden deren gesamte Informationen detailliert gespeichert. Dies beinhaltet Gruppen, Aufgaben, Teilnehmer und Termine.
\item Der Benutzer w�hlt im auftauchenden Dialog einen Dateinamen. Ein Druck auf ``Speichern'' beendet die Aktion und erstellt die CSV-Datei mit den gew�nschten Informationen.
\end{enumerate}
\paragraph{Alternativen:}
\begin{enumerate}
\item[1a.] Es gibt noch keine Veranstaltungen, Lehrenden oder Studenten. In dem Fall wird auch nichts exportiert.
\item[2a.] Der Benutzer hat kein Schreibrecht, das Speichern schl�gt fehl. In diesem Fall sollten Sie Ihre Benutzerrechte auf dem PC �berpr�fen.
\end{enumerate} 

\subsubsection{/F0011/CSV-Daten importieren}
\label{subsubsec:/F0011/}
\paragraph{Voraussetzung:}
\begin{enumerate}
\item Der Benutzer hat PrakMan gestartet und die Verbindung zu einer Datenbank erfolgreich hergestellt.
\end{enumerate}
\paragraph{Ablauf:}
\begin{enumerate}
\item Der Benutzer klickt im Baum mit der rechten Maustaste auf einen der �ste ``Lehrende'', ``Studenten'' \emph{oder} direkt auf eine bestimmte Veranstaltung und w�hlt ``Import''.

Wird ein Ast ausgew�hlt werden nur Listeninformationen der Objekte in diesen Ast �bernommen. Wird eine einzelne Veranstaltung ausgew�hlt, wird auch eine CSV-Datei mit detaillierten Informationen �ber diese Veranstaltung ben�tigt. Dies beinhaltet Gruppen, Aufgaben, Teilnehmer und Termine.
\item Der Benutzer w�hlt im auftauchenden Dialog einen Dateinamen. Ein Druck auf ``�ffnen'' beendet die Aktion und erstellt die Objekte im Baum respektive f�llt die gew�hlte Veranstaltung mit detaillierten Informationen.
\end{enumerate}
\paragraph{Alternativen:}
\begin{enumerate}
\item[1a.] Es gibt noch keine Veranstaltung, die direkt ausgew�hlt werden kann. In diesem Fall m�ssen Sie zuerst eine Veranstaltung erstellen.
\item[2a.] Ihre CSV-Datei ist falsch formatiert und wird vom System nicht verstanden.
\end{enumerate}

\subsubsection{/F0012/Ein Projekt benoten}
\label{subsubsec:/F0012/}
\paragraph{Voraussetzung:}
\begin{enumerate}
\item Der Benutzer hat PrakMan gestartet und die Verbindung zu einer Datenbank erfolgreich hergestellt.
\item Es existiert mindestens eine Veranstaltung, die zumindest ein Projekt enth�lt und deren Projekte-Ansicht ge�ffnet ist.
\end{enumerate}
\paragraph{Ablauf:}
\begin{enumerate}
\item Ein Doppelklick auf das Projekt �ffnet ein Auswahlfenster, in dem Ausgabe -und Abgabetermin festgelegt, Studenten zum Projekt hinzugef�gt und dessen Beschreibung editiert werden kann.
\item Ein Doppelklick auf das Notenfeld eines Studenten �ffnet eine Eingabemaske, in der eine neue Note f�r diesen Studenten festgelegt werden kann.
\end{enumerate}
\paragraph{Alternativen:}
Keine


\subsection{Produktdaten}
\label{subsec:Produktdaten}

\subsubsection{/D0001/Studentendaten}
\label{subsubsec:/D0001/}
Vorname, Nachname, E-Mail-Adresse und Matrikelnummer von Studenten. Des weiteren Projekte und Termine, an denen Studenten teilnehmen bzw. teilgenommen haben sowie die Benotung einzelner Projekte.

\subsubsection{/D0002/Dozentendaten}
\label{subsubsec:/D0002/}
Vorname und Nachname eines Dozenten.

\subsubsection{/D0003/Veranstaltungsdaten}
\label{subsubsec:/D0003/}
Name, Beschreibung, Teilnehmer, Projekte und Termine einer Veranstaltung.

\subsection{Produktleistungen}
\label{subsec:Produktleistungen}

\subsubsection{/L0001/Komfort}
\label{subsubsec:/L0001/}
Das Erstellen von Veranstaltungen, F�llen mit Studenten und Verteilen von Aufgaben soll m�glichst einfach und schnell m�glich sein. Das Programm soll die Arbeit erleichtern und nicht verkomplizieren.

\subsection{Benutzungsoberfl�che}
\label{subsec:Benutzungsoberflaeche}

\subsubsection{Bildschirmlayout}
\label{subsubsec:Bildschirmlayout}
Das Bildschirmlayout soll die Daten �bersichtlich darstellen und einen m�glichst schnellen und effektiven Zugriff erm�glichen. Es soll m�glich sein, mehrere Informationen gleichzeitig zu �ffnen.

\subsubsection{Drucklayout}
\label{subsubsec:Drucklayout}
Das Drucklayout soll m�glichst �bersichtlich Informationen �ber die Teilnehmer einer Veranstaltung und deren Benotung liefern.

\subsubsection{Tastaturlayout}
\label{subsubsec:Tastaturlayout}
Eine gesonderte Hotkey-Nutzung der Tastatur ist nicht vorgesehen. Das Programm soll es daf�r erm�glichen, mit der Maus m�glichst einfach von einer Information zur n�chsten damit zusammenh�ngenden zu kommen.

\subsubsection{Dialoglayout}
\label{subsubsec:Dialoglayout}
Die Dialoge sollen m�glichst viele Icons enthalten, die ihren Zweck direkt deutlich machen. Dazu sollen Tooltips existieren, die auch bei einer Unklarheit weiterhelfen.

\subsection{Qualit�tszielbestimmungen}
\label{subsec:Qualitaetszielbestimmungen}
\begin{itemize}
\item Funktionalit�t: \hfill Sehr wichtig
\item Zuverl�ssigkeit: \hfill Wichtig
\item Benutzbarkeit: \hfill Relativ wichtig
\item Effizienz: \hfill Unwichtig
\item �bertragbarkeit: \hfill Vernachl�ssigbar
\end{itemize}

\subsection{Globale Testszenarien und Testf�lle}
\label{subsec:Testszenarien}

\subsubsection{/T0001/Eingabe einer bereits vorhandenen oder negativen Matrikel-Nr.}
\label{subsubsec:/T0001/}
Die Eingabe wird mit einer Fehlermeldung zur�ckgewiesen.

\subsubsection{/T0002/Angeben einer ung�ltig formatierten CSV-Datei beim Import}
\label{subsubsec:/T0002/}
Die Datei wird als falsch formatiert zur�ckgewiesen.


\subsection{Entwicklungsumgebung}
\label{subsec:Entwicklungsumgebung}
\begin{itemize}
\item Java 1.5
\item Eclipse
\item Subversion zur verbesserten Teamarbeit
\end{itemize}

\subsection{Erg�nzungen}
\label{subsec:Ergaenzungen}
Keine Erg�nzungen.

\subsection{Begriffserkl�rung}
\label{subsec:Begriffserklaerung}
Vollst�ndige Erkl�rung wichtiger Begriffe, siehe Abschnitt \ref{chap:Glossar} (S.\pageref{chap:Glossar}).


\section{L�sungsverfahren und Algorithmen}
\label{sec:Algorithmen}
Es werden keine au�ergew�hnlichen Algorithmen verwendet.

\section{Hardware- und Softwareanforderungen}
\label{sec:Anforderungen}
Zum Betrieb des Programms erforderlich sind
\begin{itemize}
\item Java Version 1.5 oder h�her
\item F�r Mehrbenutzer-Verwendung: Eine externe Datenbank (Postgres)
\item Bei lokaler Nutzung: Speicherplatz f�r die zu haltenden Daten
\end{itemize}

\section{Ben�tigte Dateien und Directories}
\label{sec:DateienUndDirectories}
Zum Start des Programms wird allein die Datei ``prakman.jar'' ben�tigt, welche das Programm sowie alle Treiber f�r die unterst�tzten Datenbanken enth�lt.
